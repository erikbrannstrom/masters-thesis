\documentclass[a4paper]{article}

\usepackage[utf8]{inputenc}
\usepackage[T1]{fontenc}
\usepackage{natbib}

\begin{document}

\title{Correlating entities based on DSL specifications of entity group relations}
\author{Erik Brännström\\
  Chalmers University of Technology}
\date{}
\maketitle

\section{Related works}
Useful terminology is defined by \citet{IBM2006} in the field of autonomic computing. An autonomic manager is a component that
collects data from a system and, based on this data, performs actions with the purpose of improving the system. This control
loop is divided into four sub-tasks called monitor (collect system information), analyze (correlate and model data), plan
(design behaviour required to reach goal) and execute (run the planned actions), sometimes refered to as MAPE.

An interesting area in data mining is association rule learning. \citet{Agrawal1993} describe how, given a large set of
transactions containing any number of different items, rules can be identified between these items. In the case of commerce,
such rules would describe how the purchase of one product would, with a probability above a certain threshold, also infer
the purchase of another product.

\citet{Srikant1997} expands the above research by applying contraints on the items in the resulting association rules. These
constraints can either be expressed using specific items or with taxonomy rules, e.g. if an item is a descendant or ancestor
of another item.

\citet{Chen1996} give an overview of the field of data mining where they mention the aspect of multi-level data mining, which
states that correlations may not commonly exist on the lowest level of granularity, but instead by forming groups of related
items. An example given would be that a specific brand of milk does not necessarily imply the purchase of a specific brand of
bread, however purchasing milk of any kind may still be correlated to the purchase of bread irrespective of brand.

Another area in data mining is that of classifiers, and more specifically decision trees. \citet{Quinlan1986} decribe how
decision trees can be created from a training set and how well it handles the problems of unknown attributes values and noise
in the data.

\section{Foundations}

In data mining, the input to a system can be described using the terms \emph{concepts}, \emph{instances} and \emph{attributes},
where concept is the actual result of the mining, i.e. what we want to be learned; an instance is one single example of data to be
mined and can be compared to a row in a database; and attribute is a property of an instance, which in the database analogy would
be a column \citep{Witten2011}.

The Weka Project has defined an input format to be used for their open source data mining software called ARFF (Attribute-Relation
File Format).

The use of item contraints in association rule mining \citep{Srikant1997} can most likely be used for this project. The
relation between target groups and ads can, based on existing metrics, be restructured into a transactional table where each
transaction is an impression (in the case of the impression per click metrics) so that each entry would have an attribute
stating whether or not the impression yielded a click. The item constraint would then be used to only formulate association
rules that lead to clicks.

Considering that the original market basket problem assumes binary attributes and that relational data can contain both
quantative and categorical data, \citep{Srikant1996} gives a useful description on how such values can be mapped onto the
binary (or boolean) problem space.

\citet{Hahsler2007} presents a package for the R software environment used in statistical computing, which implements a base
for transaction databases as well as integration with two of the most common mining algorithms, Apriori and Eclat. One of the
clever implementation choices is the use of the sparse matrix data structure, which greatly reduces the memory load.

A thorough literature study on domain-specific languages is covered by \citet{Deursen2000} and covers high-level topics, such as
potential benefits and risks of using DSLs and exemplifying with a number of languages from different areas. The article also
mentions the phases of development in the creation of a new DSL, a topic which is described in more detail by \citet{Mernik2005}.
The latter article also provides insight into the relevant choices that need be made in the creation process, and both articles
help define a useful terminology for discussing the topic of domain-specific languages.

\section{Suggested problems}
\begin{itemize}
	\item Define a DSL for transforming a relational data structure with metrics into a transaction-like table,
			perhaps also with taxonomies for generalizing values.
	\item Modify an existing association rule mining algorithm to work directly with non-transaction-like tables.
	\item Integrate existing transaction metrics to reduce load on rule mining algorithm.
\end{itemize}

\section{Questions}
\begin{itemize}
	\item Which parts of the data should be input simultanously? Including a new campaign in a data set of long-running
		campaigns will most likely lead to a very low support for any identified rule. The problem with applying the algorithm
		on only one campaign at a time is that broader rules might not be identified.
\end{itemize}

\section{Project plan}
The proposed workflow of this thesis is inspired by the agile practices that perhaps most notably have become
common in the software development industry. The most important methods that should be applied to this work are
iterative project development and frequent ``customer collaboration'', in this case thesis suporvisors from both
Chalmers University of Technology and Duego Technologies AB.

The project is expected to run for about 20 weeks. By dividing this relatively long period of time into iterations
of 1--3 weeks, the project will be easier both to manage from my point of view as well as provide ample opportunity
for the supervisors to have their feedback incorporated.

The initial iteration will most likely require a heavy focus on research and studies of the related fields. The following
iterations will all include work on the three areas mentioned in section \ref{sec:approach} on page \pageref{sec:approach}.
Once the research has progressed to an extent that the problem space as well as the approach of the solution are well defined,
more effort can be put into the development of the final application that is to be used by Duego.

\section{Thesis outline}
Todo

\section{Original proposal}

\subsection{Introduction}
Duego was founded in 2010 with the idea of creating a social networking site that would target people who wanted to meet new
people. This stand in contrast to the existing networks that either manage people who the user already know (e.g. Facebook)
or where the user try to find a life partner (i.e. dating sites). Online advertisement is leveraged as a way to promote the
site to new users. This is done both using targeted ads that are adapted based on the target demographics on social networks,
as well as more conventional ads on websites. Today, managing these campaigns is a manual process that is outsourced to
another company. The wish of Duego is to automate this process using software. Such a system would be required to analyze
the existing campaigns with regard to ads and target groups along with campaign metrics such as clicks per impression,
conversion ratio, etcetera, and use this data to suggest, or even automatically add, new campaign ads.

\subsection{Problem}
The problem statement given in the background section is described from the view of Duego, however it can be generalized into
a much more general problem in the sphere of software engineering.

Firstly, the terms entity, entity group and metrics need to be defined. An entity is something that has distinct properties
that enables it to be compared to other entities. As such, entities do not have to be unique, and two separate entities are
further defined to be equal if they share the same properties of which the corresponding values are equal.

An entity group is the set of all entities that share one or more defined property values as given by the group definition.
In object-oriented terms, an entity group would be a class where as entities are instances of classes (i.e. objects). Metrics
is by definition a specific entity group, however for the sake of clarity I have chosen not to refer to it as such, as it is
considered such an important concept in the problem statement.

By analyzing entities that belong to separate entity groups, I wish to identify correlations between unique properties of
these with the goal of maximizing the appropriate metrics. Metrics are in turn measured for each combination of two entities
that are not of the same type. The expected input to the resulting system are entities and their metrics which should yield
an output of new combinations consisting of either already existing entities or suggested new entities.

The most prominent problem to be solved is how to design an algorithm that can efficiently calculate these correlations for
large amounts of data. The data set that will be used in production consists of tens of millions of these entities, which
means hundreds of millions of properties that should be correlated. Such data sizes can generally not be handled using
ordinary brute-force-type algorithms, but requires some form of heuristics to increase efficiency while decreasing accuracy.
Another expected requirement is to not have to run all previous data through the system each time, but rather caching results
and improving them incrementally.

\subsection{Relevance}
The requirements discussed in the problem description are closely related to both the field of machine learning as well as
that of algorithms and algorithm design, meaning that the final product will most likely have to incorporate strategies from
both fields.

\subsection{Approach}
\label{sec:approach}
The general problem can be divided into three subtasks that need to be completed in order to create a system that fulfill the
defined requirements.
\begin{itemize}
	\item Create a domain specific language (DSL) for specifying the desired entity groups, their relations and which
	properties to optimize
	\item Design algorithm for optimizing the defined task which can be proven to provide output of high quality
	\item Extend system with machine learning methods to increase data handling capabilities without causing output quality to
	 drop below reasonable thresholds
\end{itemize}
Since the general solution to the problem will simultaneously be applied to a specific problem area (described in the
background), the definitions of quality are based on customer expectations. As such, it is recommended that the thesis work is
performed using an agile approach to maximize the impact of input from supervisors both at Duego as well as Chalmers University
of Technology.

\subsection{Verification}
The result of the thesis work would be verified by applying the results to the problem suggested by Duego in a real-world setting.
 The suggested advertisement and target groups from the system will be added to Facebook which will allow us to analyze the
 effectiveness of the software. This can be done continuously over the course of the development in addition to more traditional
 software unit and system testing.

\bibliographystyle{plainnat}
\bibliography{references} % references.bib
\end{document}